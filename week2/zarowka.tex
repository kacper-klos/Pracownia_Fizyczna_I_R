\documentclass[12pt]{article}

\usepackage[utf8]{inputenc}
\usepackage[T1]{fontenc}
\usepackage{lmodern}
\usepackage{float}
\usepackage[polish,provide=*]{babel}
\usepackage{amsmath}
\usepackage{latexsym,amsfonts,amssymb,amsthm,amsmath}
\usepackage{enumitem}
\usepackage{hyperref}
\usepackage{graphicx}
\usepackage{subcaption}
\usepackage{booktabs}
\graphicspath{{./images/}}

\setlength{\parindent}{0in}
\setlength{\oddsidemargin}{0in}
\setlength{\textwidth}{6.5in}
\setlength{\textheight}{8.8in}
\setlength{\topmargin}{0in}
\setlength{\headheight}{18pt}

\title{Pomiar Masy Włókna Żarówki}
\author{Kacper Kłos}

\begin{document}

\maketitle
W tym raporcie wyznaczyliśmy masę włókna małej żarówki wolframowej. Otrzymana przez nas wartość wynosi $m = (0{,}69 \pm 0{,}04) \, mg$. Do otrzymania tego wyniku wyznaczyliśmy opór żarówki $R_0 = (2{,}887 \pm 0{,}015) \, \Omega$ oraz opornika pomocnicznego $R = (9{,}795 \pm 0{,}015) \, \Omega$ korzystając z miernika uniwersalnego. Następnie podłączając układ do stałego źródła prądu, wyznaczyliśmy zależność liniową dla pomiarów mocy rozpraszanej przez żarówkę w zależności od jej oporu dla małych temperatur ($P_w = a R_w + b$) i otrzymaliśmy $a = (0{,}00383 \pm 0{,}00008) \, W \Omega^{-1}$ oraz $b = (-0{,}0121 \pm 0{,}0005) \, W$. Ostatecznie dla kilku napięć zasilacza, wyznaczyliśmy pochodną napięcia na oporniku pomocniczym tuż po zamknięciu układu poprzez dopasowanie krzywej do danych odczytanych na oscyloskopie. Łącząc wszystkie wspomniane wartości, otrzymaliśmy punkty pomiarowe do równania do którego dopasowana krzywa wyznaczyła nam iloczyn masy i ciepła właściwego włókna na $mc_w=(1{,}00 \pm 0{,}05) \times 10^{-5}\, J/K$. Za pomocą tego wyniku i danych o cieple właściwym wolframu otrzymaliśmy masę włókna $m = (0{,}69 \pm 0{,}04) \, mg$.
\newpage

\section{Wstęp}
W niniejszym doświadczeniu przedstawiamy metodę wyznaczania masy włókna żarówki wolframowej z wykorzystaniem sprzętu dostępnego w pracowni elektronicznej: multimetru, zasilacza, oscyloskopu oraz rezystora.  
Wyznaczymy najpierw tempo zmiany temperatury wolframu w badanej żarówce, a także moc dostarczaną i wytracaną przez żarówkę.  
Na podstawie tych danych dopasujemy prostą zależność, która – w połączeniu z wartościami materiałowymi wolframu (zaczerpniętymi z tabel) – pozwoli nam oszacować masę włókna.

\section{Podstawy Teoretyczne}
W tej części przedstawimy częściowe wyprowadzenie wzorów kluczowych w doświadczeniu, przedstawone wzory jak i całe wyprowadzenie można znaleźć \cite{skrypt}.

W trakcie całego doświadczenia przyjmujemy, że rezystancje wszystkich elementów obwodu, poza żarówką, są stałe.  
Dla żarówki zakładamy, iż jej opór zmienia się liniowo z temperaturą w badanym zakresie i można go wyrazić następującym wzorem:
\begin{equation}
    R_w(T) = R_0 \bigl(1+\alpha\,(T - T_0)\bigr),
    \label{eq:resistance}
\end{equation}
gdzie:
\begin{itemize}
\item $R_w(T)$ – opór włókna przy temperaturze $T$,  
\item $R_0$ – opór włókna w ustalonej temperaturze referencyjnej $T_0$,  
\item $\alpha$ – współczynnik temperaturowy oporu (stała proporcjonalności dla wolframu).
\end{itemize}

Prąd przepływający przez żarówkę dostarcza do niej moc, która jest wykorzystywana na:  
\begin{enumerate}
\item ogrzanie włókna (zwiększanie jego temperatury),  
\item straty ciepła (głównie w postaci wypromieniowanej energii).  
\end{enumerate}
Możemy zatem zapisać:
\begin{equation}
    P = P_s(T) + m\,c_w\,\frac{\Delta T}{\Delta t},
    \label{eq:mass}
\end{equation}
gdzie:  
\begin{itemize}
\item $P$ – moc dostarczana żarówce,  
\item $P_s(T)$ – moc wytracana (odprowadzana do otoczenia) przez żarówkę,  
\item $m$ – masa włókna,  
\item $c_w$ – ciepło właściwe wolframu,  
\item $\frac{\Delta T}{\Delta t}$ – tempo zmiany temperatury włókna w czasie.  
\end{itemize}

Zakładamy, że moc tracona przez żarówkę $P_s$ zależy tylko od temperatury włókna. Ze względu na to, że opór żarówki zależy liniowo od temperatury (zgodnie z równaniem \eqref{eq:resistance}), moc rozpraszana na żarówce jest wprost proporcjonalna do jej rezystancji.

Ponieważ znane są wartości tablicowe współczynników $c_w$ oraz $\alpha$ dla wolframu, do wyznaczenia masy włókna potrzebujemy doświadczalnie określić:  
\begin{itemize}
\item moc $P$ dostarczaną do żarówki,  
\item moc $P_s$ traconą przez włókno,  
\item zmianę temperatury w czasie $\frac{\Delta T}{\Delta t}$.  
\end{itemize}
Ostatecznie, korzystając z równania \eqref{eq:mass}, znając wymienione powyżej parametry dla kilku punktów pomiarowych będziemy w stanie dopasować linię aby wyznaczyć $mc_w$.

\section{Układ Doświadczalny}
\subsection{Pomiar Stacjonarny}
W pierwszej części badamy zależność mocy oddawanej przez żarówkę od jej oporu. W tym celu korzystamy z układu z rys. \ref{fig:pomiar_stac}:
\begin{figure}[H]
    \centering
    \includegraphics[scale=0.25]{static}
    \caption{Diagram układu używanego do pomiarów statycznych (Strona wykorzystana do narysowania diagramu: \cite{diagram}).}
    \label{fig:pomiar_stac}
\end{figure}

Rozpisując równania na prądy i napięcia w tym układzie, otrzymujemy:

\begin{equation}
    P_s = P = U_z \cdot I = (U_0 - U_R) \frac{U_R}{R},
    \label{eq:power_dis}
\end{equation}
\begin{equation}
    R_w = \frac{U_0 - U_R}{U_R}\,R.
    \label{eq:bulb_resistance}
\end{equation}

Opory $R$ oraz $R_w$ łatwo zmierzyć przy pomocy multimetru. Z kolei do określenia $P_s$ wykorzystujemy fakt, że w stanie ustalonym (gdy temperatura przestaje się wyraźnie zmieniać,
a napięcie na żarówce stabilizuje się) moc dostarczana musi równać się mocy wytracanej, tj. $P_s = P$ zgodnie z \eqref{eq:power_dis}.

Dla niewielkich wartości oporu włókna (a tym samym małych wartości temperatury) zależność $P_s(R_w)$ jest w dobrym przybliżeniu liniowa, co motywuje dopasowanie funkcji:
\begin{equation}
    P_s = a\,R_w + b.
    \label{eq:power_line}
\end{equation}

\subsection{Pomiary Dynamiczne}
W części dynamicznej układ jest analogiczny do użytego w pomiarach stacjonarnych, z tą różnicą, że zamiast woltomierza zastosowano oscyloskop (rys. \ref{fig:pomiar_dyn}):
\begin{figure}[H]
    \centering
    \includegraphics[scale=0.25]{dynamic}
    \caption{Diagram układu używanego do pomiarów dynamicznych (źródło: \cite{diagram}).}
    \label{fig:pomiar_dyn}
\end{figure}

Korzystając z równania \eqref{eq:resistance}, zauważamy, że przy zmianach oporu w czasie:
\begin{equation}
    \frac{\Delta T}{\Delta t} = \frac{1}{\alpha\,R_0}\,\frac{\Delta R_w}{\Delta t}.
    \label{eq:temp_der}
\end{equation}
Dla bardzo małych przedziałów czasowych $\frac{\Delta R_w}{\Delta t}$ możemy traktować jako pochodną $d R_w / d t$. Wówczas z \eqref{eq:bulb_resistance} wynika:
\begin{equation}
    \frac{d R_w}{d t} = -\,R\,\frac{U_0(t)}{U_R(t)^2}\,\frac{d U_R}{dt}.
    \label{eq:bulb_der}
\end{equation}
Przyjmując, że w krótkim czasie po przyłożeniu napięcia generatora $U_0(t)$ nie zmienia się istotnie (napięcie zasilania jest niemal stałe), możemy zastąpić pochodne różnicami i zapisać:
\begin{equation}
    \frac{\Delta T}{\Delta t} = -\,\frac{R\,U_0(t_0)}{\alpha\,R_0\,U_R(t_0)^2}\,\frac{\Delta U_R}{\Delta t},
    \label{eq:temp_delta}
\end{equation}
gdzie $t_0$ oznacza moment, w którym napięcie ustabilizowało się i zaczyna zachowywać się niemal liniowo.

Łącząc równania \eqref{eq:power_dis}, \eqref{eq:bulb_resistance}, \eqref{eq:temp_delta} z \eqref{eq:mass}, otrzymujemy ostatecznie:
\begin{equation}
    (U_0(t_0) - U_R(t_0))\frac{U_R(t_0)}{R}
    \;-\;
    a\,\frac{U_0(t_0) - U_R(t_0)}{U_R(t_0)}\,R
    \;-\; b
    \;=\;
    -\,m\,c_w\,\frac{R\,U_0(t_0)}{\alpha\,R_0\,U_R(t_0)^2}\,\frac{\Delta U_R}{\Delta t}.
    \label{eq:final}
\end{equation}
Dopasowując krzywą do powyższego równania, możemy wyznaczyć masę $m$ włókna.

\section{Wyniki Pomiarów}
W tej części przedstawiamy dane pomiarowe wraz z analizą niepewności.  
Korzystamy z następujących oznaczeń:  
\[
    \bar{x} \ \text{- średnia z wartości }x,\quad s_x \ \text{- niepewność statystyczna,}
\]
\[
    \delta x \ \text{- niepewność pomiarowa,}\quad u(x) \ \text{- niepewność całkowita.}
\]
Często też posługujemy się standardowym wzorem na propagację niepewności dla funkcji wielu zmiennych $f(x_1,\ldots,x_i)$:
\begin{equation}
    \delta f(x_1,\ldots,x_i)
    = \sqrt{\left(\frac{\partial f}{\partial x_1}\,\delta x_1\right)^2 + \dots + \left(\frac{\partial f}{\partial x_i}\,\delta x_i\right)^2}.
    \label{eq:error_propagation}
\end{equation}

\subsection{Opory}
Z instrukcji multimetru \cite{multimeter} wynika, że dla oporów w zakresie poniżej $200\,\Omega$ niepewność pomiarowa wynosi $\delta R =0.01\,\Omega + 3R\cdot10^{-4}$ gdzie R to zmierzony opór.
W oparciu o tą wiedze i wyniki pomiarów znajdujące się w tabeli \ref{tab:opory} wyznaczamy opory.
\begin{table}[H]
    \centering
    \begin{tabular}{c|cc}
        \toprule
        \textbf{Nr} & $R \, [\Omega]$ & $R_0 \, [\Omega]$ \\
        \midrule
        1 & 9{,}802  & 2{,}870 \\
        2 & 9{,}786  & 2{,}888 \\
        3 & 9{,}796  & 2{,}902 \\
        \bottomrule
    \end{tabular}
    \caption{Pomiary oporów rezystora ($R$) oraz żarówki w temperaturze referencyjnej ($R_0$).}
    \label{tab:opory}
\end{table}

Przyjmując wartości rzeczywiste jako średnie z pomiarów, otrzymujemy:
\[
    R = 9{,}795\,\Omega, \quad s_{R} = 0{,}0047\,\Omega, \quad \delta R = 0{,}0130\,\Omega, \quad u(R) = 0{,}014\,\Omega,
\]
\[
    R_0 = 2{,}887\,\Omega, \quad s_{R_0} = 0{,}0093\,\Omega, \quad \delta R_0 = 0{,}0109\,\Omega, \quad u(R_0) = 0{,}015\,\Omega,
\]

\newpage

\subsection{Pomiary Stacjonarne}
Dla wszystkich zarejestrowanych pomiarów przyjmujemy niepewność zasilacza według instrukcji \cite{generator}, tj. $0{,}05\% + 10\,\mathrm{mV}$.  
Za niepewność pomiaru woltomierza uznajemy maksymalne wahanie napięcia po jego ustabilizowaniu. Tabela \ref{tab:resistor_voltage} przedstawia uzyskane wyniki.
\begin{table}[H]
    \centering
    \begin{tabular}{c|cccc}
        \toprule
        Nr & $U_0 \,[V]$ & $U_R \,[V]$ & $\delta U_0 \,[V]$ & $\delta U_R \,[V]$ \\
        \midrule
        1  & 0{,}11  & 0{,}0805 & 0{,}011 & 0{,}0001  \\
        2  & 0{,}21  & 0{,}1547 & 0{,}011 & 0{,}0001  \\
        3  & 0{,}31  & 0{,}2255 & 0{,}011 & 0{,}0001  \\
        4  & 0{,}41  & 0{,}2907 & 0{,}011 & 0{,}0001  \\
        5  & 0{,}51  & 0{,}3472 & 0{,}011 & 0{,}0001  \\
        6  & 0{,}61  & 0{,}3921 & 0{,}011 & 0{,}0001  \\
        7  & 0{,}71  & 0{,}4284 & 0{,}011 & 0{,}0001  \\
        8  & 0{,}81  & 0{,}4618 & 0{,}011 & 0{,}0001  \\
        9  & 0{,}91  & 0{,}4944 & 0{,}011 & 0{,}0001  \\
        10 & 1{,}01  & 0{,}5270 & 0{,}011 & 0{,}0001  \\
        11 & 1{,}11  & 0{,}5592 & 0{,}011 & 0{,}0001  \\
        12 & 1{,}21  & 0{,}5908 & 0{,}011 & 0{,}0001  \\
        13 & 1{,}31  & 0{,}6221 & 0{,}011 & 0{,}0001  \\
        14 & 1{,}41  & 0{,}6524 & 0{,}011 & 0{,}0001  \\
        15 & 1{,}51  & 0{,}6823 & 0{,}011 & 0{,}0001  \\
        16 & 1{,}76  & 0{,}7548 & 0{,}011 & 0{,}0010  \\
        17 & 2{,}01  & 0{,}8228 & 0{,}012 & 0{,}0010  \\
        18 & 2{,}25  & 0{,}8881 & 0{,}012 & 0{,}0010  \\
        19 & 2{,}50  & 0{,}9512 & 0{,}012 & 0{,}0010  \\
        20 & 3{,}01  & 1{,}0717 & 0{,}012 & 0{,}0010  \\
        21 & 3{,}51  & 1{,}1843 & 0{,}012 & 0{,}0010  \\
        22 & 4{,}51  & 1{,}3915 & 0{,}013 & 0{,}0010  \\
        23 & 5{,}51  & 1{,}5800 & 0{,}013 & 0{,}0010  \\
        24 & 7{,}50  & 1{,}9144 & 0{,}014 & 0{,}0020  \\
        25 & 9{,}51  & 2{,}2113 & 0{,}015 & 0{,}0020  \\
        \bottomrule
    \end{tabular}
    \caption{Pomiary zależności napięcia $U_R$ na rezystorze od napięcia zasilającego $U_0$ oraz ich błędy pomiarowe.}
    \label{tab:resistor_voltage}
\end{table}

Korzystając z równań \eqref{eq:bulb_resistance} i \eqref{eq:power_dis}, obliczamy niepewności $\delta R_w$ i $\delta P_s$ na drodze propagacji (zgodnie z \eqref{eq:error_propagation}).

\newpage

Wykres zależności mocy rozpraszanej przez żarówkę od jej oporu (z naniesionymi niepewnościami) przedstawiono na rys. \ref{fig:power_res_full}.
\begin{figure}[H]
    \centering
    \includegraphics[scale=0.58]{pomiary_moc}
    \caption{Zależność mocy rozpraszanej przez żarówkę od jej oporu $R_w$, wraz z niepewnościami pomiarowymi.}
    \label{fig:power_res_full}
\end{figure}

Dla małych wartości oporu (przedział od ok. 1\,\(\Omega\) do 7\,\(\Omega\), czyli pomiary od 2 do 8) zależność ta jest prawie liniowa. W tym zakresie dopasowujemy funkcję \eqref{eq:power_line} metodą regresji ortogonalnej (rys. \ref{fig:power_res_line}).

\begin{figure}[H]
    \centering
    \includegraphics[scale=0.58]{pomiary_moc_linia}
    \caption{Dopasowanie liniowe do zależności $P_s(R_w)$ dla przedziału $R_w \in [1\,\Omega,\; 7\,\Omega]$.}
    \label{fig:power_res_line}
\end{figure}
Wyniki dopasowania to:
\[
    a = 0{,}00383 \ W\Omega^{-1}, \quad u(a) = 0{,}00008 \ W\Omega^{-1}, \qquad
    b = -\,0{,}0121 \ W, \quad u(b) = 0{,}0005 \ W.
\]
Na tej podstawie można oszacować opór spoczynkowy żarówki, przyjmując, że w chwili $P_s = 0$ (brak emisji energii) $R_w = R_0$:
\begin{equation}
    R_0 = -\frac{b}{a}, 
    \quad
    \delta R_0^2 = \left(\frac{u(a)}{b}\right)^2 + \left(\frac{a}{b^2}u(b)\right)^2.
\end{equation}
Podstawiając dane, otrzymujemy:
\[
    R_0 = 3{,}14\,\Omega, \quad u(R_0) = 0{,}13\,\Omega.
\]
Jest to wartość nieco odbiegająca od rezultatu otrzymanego bezpośrednio z pomiarów multimetrem (różnica <\,10\%). 
W dalszej analizie korzystamy jednak z wartości pomierzonej multimetrem, ze względu na jej mniejszą niepewność oraz bardziej zaufaną metodologię (trudno przewidzieć co dzieje się przey ciągłym zmniejszaniu oporu włókna $R_w$).

\subsection{Pomiary Dynamiczne}
W tej części badamy zależność $U_R(t)$ tuż po przyłożeniu napięcia $U_0$. Na oscyloskopie rejestrujemy krótki przedział czasowy zaraz po „skoku” napięcia, dopasowujemy prostą do $U_R(t)$ i wyznaczamy jej współczynnik kierunkowy (pochodną). W tabeli \ref{tab:lines_data} zestawiono zarejestrowane napięcia w seriach pomiarowych, podczas gdy w suplemencie znajdują się wykresy \ref{fig:line_fit_graph} z dopasowaniymi krzywymi.

\begin{table}[H]
    \centering
    \begin{tabular}{c|c|ccccc|c}
        \toprule
        $U_0 \,[V]$ & $\delta U_0 \,[V]$ & \multicolumn{5}{c}{Serie Pomiarowe} & $\delta U_R \,[V]$ \\
        \midrule
        & & \multicolumn{5}{c}{$U_R \,[V]$} & \\
        10{,}1  & 0{,}016 & 6{,}760  & 6{,}740  & 6{,}680  & 6{,}640  & 6{,}600 & 0{,}12 \\
                &         & \textit{3{,}040}  & \textit{3{,}240}  & \textit{3{,}560}  & \textit{3{,}920}  & \textit{4{,}080} &   \\
                &         & \multicolumn{5}{c}{\textbf{t [ms]}} & \\[6pt]
        \midrule
        9{,}01  & 0{,}015 & 6{,}130  & 6{,}070  & 6{,}010  & 5{,}970  & 5{,}930 & 0{,}12 \\
                &         & \textit{7{,}160}  & \textit{7{,}800}  & \textit{8{,}280}  & \textit{8{,}760}  & \textit{9{,}120} & \\
                &         & \multicolumn{5}{c}{\textbf{t [ms]}} & \\[6pt]
        \midrule
        8{,}00  & 0{,}014 & 5{,}600  & 5{,}540  & 5{,}500  & 5{,}420  & 5{,}360 & 0{,}10 \\
                &         & \textit{5{,}600}  & \textit{6{,}400}  & \textit{7{,}000}  & \textit{8{,}000}  & \textit{9{,}000} & \\
                &         & \multicolumn{5}{c}{\textbf{t [ms]}} & \\[6pt]
        \midrule
        7{,}01  & 0{,}014 & 4{,}600  & 4{,}560  & 4{,}520  & 4{,}480  & 4{,}440 & 0{,}08 \\
                &         & \textit{7{,}400}  & \textit{8{,}200}  & \textit{9{,}000}  & \textit{9{,}800}  & \textit{10{,}60} & \\
                &         & \multicolumn{5}{c}{\textbf{t [ms]}} & \\[6pt]
        \midrule
        6{,}01  & 0{,}014 & 4{,}330  & 4{,}270  & 4{,}230  & 4{,}190  & 4{,}130 & 0{,}08 \\
                &         & \textit{8{,}400}  & \textit{10{,}00}  & \textit{11{,}60}  & \textit{12{,}80}  & \textit{14{,}80} & \\
                &         & \multicolumn{5}{c}{\textbf{t [ms]}} & \\[6pt]
        \midrule
        5{,}01  & 0{,}013 & 3{,}170  & 3{,}130  & 3{,}110  & 3{,}090  & 3{,}030 & 0{,}06 \\
                &         & \textit{10{,}00}  & \textit{11{,}60}  & \textit{13{,}20}  & \textit{14{,}80}  & \textit{16{,}40} & \\
                &         & \multicolumn{5}{c}{\textbf{t [ms]}} & \\[6pt]
        \midrule
        4{,}01  & 0{,}013 & 3{,}000  & 2{,}980  & 2{,}960  & 2{,}880  & 2{,}840 & 0{,}05 \\
                &         & \textit{12{,}00}  & \textit{14{,}40}  & \textit{16{,}40}  & \textit{24{,}80}  & \textit{30{,}00} & \\
                &         & \multicolumn{5}{c}{\textbf{t [ms]}} & \\[6pt]
        \midrule
        3{,}01  & 0{,}012 & 1{,}844  & 1{,}830  & 1{,}816  & 1{,}804  & 1{,}792 & 0{,}02 \\
                &         & \textit{20{,}40}  & \textit{23{,}60}  & \textit{26{,}00}  & \textit{29{,}20}  & \textit{32{,}80} & \\
                &         & \multicolumn{5}{c}{\textbf{t [ms]}} & \\[6pt]
        \midrule
        2{,}01  & 0{,}012 & 1{,}418  & 1{,}410  & 1{,}390  & 1{,}386  & 1{,}383 & 0{,}01 \\
                &         & \textit{26{,}40}  & \textit{31{,}20}  & \textit{38{,}80}  & \textit{42{,}80}  & \textit{46{,}00} & \\
                &         & \multicolumn{5}{c}{\textbf{t [ms]}} & \\[6pt]
        \bottomrule
    \end{tabular}
    \caption{Serie pomiarowe napięcia na rezystorze $U_R(t)$ od czasu $t$ tuż po zniknięciu zakłóceń, przy napięcia na zasilaczu $U_0$}
    \label{tab:lines_data}
\end{table}

Po dopasowaniu prostych metodą najmniejszych kwadratów do każdej serii otrzymujemy współczynnik kierunkowy $\frac{dU_R}{dt}$ oraz wartość $U_R(t_0)$ (pierwszy punkt po ustaleniu napięcia). Dane zestawiono w tabeli \ref{tab:dynamic_data}:

\begin{table}[H]
    \centering
    \begin{tabular}{cc|cc|cc}
        \toprule
        $U_0(t_0)\,[V]$ & $\delta U_0(t_0)\,[V]$ & $U_R(t_0)\,[V]$ & $\delta U_R(t_0)\,[V]$ & $\frac{dU_R}{dt}\,[V/ms]$ & $\delta\bigl(\frac{dU_R}{dt}\bigr)$ \\
        \midrule
        10{,}1  & 0{,}016 & 6{,}760  & 0{,}12   & -0{,}151   & 0{,}0010 \\
        9{,}01  & 0{,}015 & 6{,}130  & 0{,}12   & -0{,}102   & 0{,}0040 \\
        8{,}00  & 0{,}014 & 5{,}600  & 0{,}10   & -0{,}0713  & 0{,}0019 \\
        7{,}01  & 0{,}014 & 4{,}600  & 0{,}08   & -0{,}050   & 0{,}0000 \\
        6{,}01  & 0{,}014 & 4{,}330  & 0{,}08   & -0{,}031   & 0{,}0010 \\
        5{,}01  & 0{,}013 & 3{,}170  & 0{,}06   & -0{,}020   & 0{,}0030 \\
        4{,}01  & 0{,}013 & 3{,}000  & 0{,}05   & -0{,}0090  & 0{,}0003 \\
        3{,}01  & 0{,}012 & 1{,}844  & 0{,}02   & -0{,}0043  & 0{,}0003 \\
        2{,}01  & 0{,}012 & 1{,}418  & 0{,}01   & -0{,}00190 & 0{,}00019 \\
        \bottomrule
    \end{tabular}
    \caption{Zestawienie napięć na zasilaczu $U_0(t_0)$ i rezystorze $U_R(t_0)$ orza współczynników kierunkowych $\frac{dU_R}{dt}$ po dopasowaniu do serii pomiarowych.}
    \label{tab:dynamic_data}
\end{table}
Za niepewność każdego $U_R(t_0)$ przyjmujemy największe zaobserwowane wychylenie napięcia w danej serii. Zakładamy, że czas (odczyt z oscyloskopu) jest zmierzony z pomijalną niepewnością.

\section{Masa Włókna}
Dla wygody definiujemy zmienne pomocnicze, wynikające z równania \eqref{eq:final}:
\begin{equation}
    X = -\,\frac{R\,U_0(t_0)}{\alpha\,R_0\,U_R(t_0)^2}\,\frac{\Delta U_R}{\Delta t},
\end{equation}
\begin{equation}
    Y = \;(U_0(t_0) - U_R(t_0))\frac{U_R(t_0)}{R}
    \;-\;a\,\frac{U_0(t_0) - U_R(t_0)}{U_R(t_0)}\,R
    \;-\; b.
\end{equation}
Wówczas równanie \eqref{eq:final} przyjmuje postać:
\begin{equation}
    Y = m\,c_w\,X.
    \label{eq:final_line}
\end{equation}
Do obliczeń potrzebujemy jeszcze wartości $\alpha$ dla wolframu. Korzystamy z danych \cite{heat_resist}, przyjmując $\alpha = 4{,}5 \times 10^{-3}\,K^{-1}$.

Na rys. \ref{fig:final_graph} przedstawiono punktową zależność $Y(X)$ wraz z dopasowaną linią prostą.
\begin{figure}[H]
    \centering
    \includegraphics[scale=0.7]{final_graph}
    \caption{Zależność $Y$ od $X$ z równania \eqref{eq:final_line}, wraz z liniowym dopasowaniem do danych.}
    \label{fig:final_graph}
\end{figure}
Wynik dopasowania daje:
\[
    m\,c_w = 1{,}00 \times 10^{-5}\,\mathrm{J/K}, 
    \quad
    u(m\,c_w) = 0{,}05\times 10^{-5}\,\mathrm{J/K}.
\]
Wykorzystując wartość ciepła właściwego wolframu $c_w = 144\,\mathrm{J/(kg\,K)}$ \cite{skrypt}, otrzymujemy ostatecznie:
\[
    m =0{,}69\,\mathrm{mg},
    \quad
    u(m) = 0{,}04\,\mathrm{mg}.
\]

\newpage

\section{Podsumowanie}
W przedstawionym doświadczeniu staraliśmy się wyznaczyć masę włókna żarówki wolframowej. Na początku określiliśmy opór żarówki oraz rezystora, wykonując kilka pomiarów miernikiem uniwersalnym (tabela~\ref{tab:opory}). Otrzymaliśmy wartości $R_{0} = (2{,}887 \pm 0{,}015)\,\Omega$ dla żarówki oraz $R = (9{,}795 \pm 0{,}014)\,\Omega$ dla rezystora.

Potem zmierzyliśmy napięcie na oporniku w zależności od napięcia zasilacza, aby dopasować zależność liniową dla małych napięć do wzoru~\ref{eq:power_dis}. Z tabeli~\ref{tab:resistor_voltage} wybraliśmy kilka pierwszych punktów, w których obserwowaliśmy zależność liniową, i dopasowaliśmy krzywą widoczną na rys.~\ref{fig:power_res_line}, uzyskując wartości współczynników $a=(0{,}00383 \pm 0{,}00008)\,W\,\Omega^{-1}$ oraz $b=(-0{,}0121 \pm 0{,}0005)\,W$.

Na dalszym etapie, dopasowaliśmy prostą (tabela~\ref{tab:lines_data}) do napięcia na rezystorze w funkcji czasu dla kilku wartości napięcia ustawionych na generatorze (tabela~\ref{tab:dynamic_data}), tuż po zamknięciu układu. Wszystkie otrzymane wielkości wprowadziliśmy do równania~\eqref{eq:final}, do którego również dopasowaliśmy prostą pozwalającą nam wyznaczyć końcowy wynik masy włókna żarówki: $m=(0{,}69\pm0{,}04)\,\mathrm{mg}$. Otrzymany wynik wydaje się prawdopodobny, badana żarówka była niewielka oraz składała się z pojedyńczego drutu wolframu grubości rzędu części dziesiątych milimetra.

Błąd jaki otrzymaliśmy w znaczącej części pochodzi od wyznaczonych pochodnych, które zawierały perturbacje sygnału spowodowane nagłym zamknięciem układu. Widać to na wykresach w suplemencie \ref{fig:line_fit_graph}, oraz na końcowym wykresie \ref{fig:final_graph} na którym widoczny jest błąd zmiennej X, w przeciwieństwie do tego pochodzącego od zmiennej Y. Inne błędy napotkane w równaniach są marginalne, opór żarówki i opornika mają kolejno około $0{,}5 \%$ i $0{,}2 \%$, podobnie współczynniki znalezione w części stacjonarnej, procentowo błędy prezentują się $2\%$ dla $a$ oraz $4\%$ dla $b$. Widzimy że błędy z częsci dynamicznej musiały zawyżyć błąd masy, osiągający prawie $6\%$.


\newpage
\begin{thebibliography}{6}

\bibitem{skrypt}
Żarówka (pomiar masy włókna żarówki), Uniwersytet Warszawski, A. Łopion, P. Węgorzny, P. Fita.

\bibitem{diagram}
\url{https://www.smartdraw.com/circuit-diagram/}, tworzenie diagramów układów elektrycznych.

\bibitem{multimeter}
\url{http://pracownie1.fuw.edu.pl/przyrzady/Multimetr_Rigol_DM3058_UserGuide_EN.pdf}, instrukcja multimetru Rigol DM3058.

\bibitem{generator}
\url{http://pracownie1.fuw.edu.pl/przyrzady/Zasilacz_Rigol_DP800_UserGuide_EN.pdf}, instrukcja zasilacza Rigol DP832.

\bibitem{heat_resist}
Giancoli, Douglas C., \emph{Physics}, 4th Ed.

\end{thebibliography}

\newpage

\section{Suplement}
Wykresy dopasowań liniowych do danych z tabeli~\ref{tab:lines_data}.

\begin{figure}[H]
    \centering
    \includegraphics[scale=0.58]{nachylenie_1}
    \caption{Punkty pomiarowe oraz dopasowanie liniowe dla napięcia na generatorze $U_0 = 10{,}1 \, V$}
    \label{fig:line_fit_graph}
\end{figure}
\begin{figure}[H]
    \centering
    \includegraphics[scale=0.58]{nachylenie_2}
    \caption{Punkty pomiarowe oraz dopasowanie liniowe dla napięcia na generatorze $U_0 = 9{,}01 \, V$}
\end{figure}
\begin{figure}[H]
    \centering
    \includegraphics[scale=0.58]{nachylenie_3}
    \caption{Punkty pomiarowe oraz dopasowanie liniowe dla napięcia na generatorze $U_0 = 8{,}00 \, V$}
\end{figure}
\begin{figure}[H]
    \centering
    \includegraphics[scale=0.58]{nachylenie_4}
    \caption{Punkty pomiarowe oraz dopasowanie liniowe dla napięcia na generatorze $U_0 = 7{,}01 \, V$}
\end{figure}
\begin{figure}[H]
    \centering
    \includegraphics[scale=0.58]{nachylenie_5}
    \caption{Punkty pomiarowe oraz dopasowanie liniowe dla napięcia na generatorze $U_0 = 6{,}01 \, V$}
\end{figure}
\begin{figure}[H]
    \centering
    \includegraphics[scale=0.58]{nachylenie_6}
    \caption{Punkty pomiarowe oraz dopasowanie liniowe dla napięcia na generatorze $U_0 = 5{,}01 \, V$}
\end{figure}
\begin{figure}[H]
    \centering
    \includegraphics[scale=0.58]{nachylenie_7}
    \caption{Punkty pomiarowe oraz dopasowanie liniowe dla napięcia na generatorze $U_0 = 4{,}01 \, V$}
\end{figure}
\begin{figure}[H]
    \centering
    \includegraphics[scale=0.58]{nachylenie_8}
    \caption{Punkty pomiarowe oraz dopasowanie liniowe dla napięcia na generatorze $U_0 = 3{,}01 \, V$}
\end{figure}
\begin{figure}[H]
    \centering
    \includegraphics[scale=0.58]{nachylenie_9}
    \caption{Punkty pomiarowe oraz dopasowanie liniowe dla napięcia na generatorze $U_0 = 2{,}01 \, V$}
\end{figure}

\end{document}

